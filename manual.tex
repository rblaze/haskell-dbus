
% Copyright (C) 2009 John Millikin <jmillikin@gmail.com>
% 
% This document is distributed under the terms of the Creative Commons
% Attribution Share Alike 3.0 license. See
% <http://creativecommons.org/licenses/by-sa/3.0/>.

\documentclass[12pt]{article}

\usepackage{color}
\usepackage{hyperref}
\usepackage{booktabs}
\usepackage{multirow}
\usepackage{noweb}
\usepackage{url}
\usepackage{multicol}

% Smaller margins
\usepackage[left=1.5cm,top=2cm,right=1.5cm,nohead,nofoot]{geometry}

% Remove boxes from hyperlinks
\hypersetup{
    colorlinks,
    linkcolor=blue,
}

\makeindex

\begin{document}

\addcontentsline{toc}{section}{Contents}
\tableofcontents

\section{Introduction}

D-Bus is a low-latency, asynchronous IPC protocol. It is primarily used on
Linux, BSD, and other free UNIX-like systems. More information is available
at \url{http://dbus.freedesktop.org/}.

This package is an implementation of the D-Bus protocol. It is intended
for use in either a client or server, though currently only the client
portion of connection establishment is implemented. Additionally, it
implements the introspection file format.

\section{The message bus}

While it is possible to establish a connection between any two applications,
D-Bus is most commonly used with a central bus. This bus authorises new
connections, dispatches messages, broadcasts signals, and other ancilliary
functions.

\subsection{Connecting}

Computations for connecting to a message bus are available in the
{\tt DBus.Bus} module. Additionally, pre-defined computations exist for the
standard {\sc system}, {\sc session}, and {\sc starter} buses.

All bus connection computations return the opened connection, and the
unique namd assigned by the bus to this client.

\begin{verbatim}
getBus        ::  Address  -> IO (Connection, BusName)
getFirstBus   :: [Address] -> IO (Connection, BusName)
getSystemBus  :: IO (Connection, BusName)
getSessionBus :: IO (Connection, BusName)
getStarterBus :: IO (Connection, BusName)
\end{verbatim}

\subsection{Sending and receiving messages}

Once a connection has been established, messages may be sent or received
from the bus. Received messages will be populated with their serial and
origin, if available.

If an IO error occurs, an exception will be raised. If the bytes received
are invalid or for a different D-Bus version, a {\tt Left} value will be
returned. See the generated API index for a list of possible error
conditions.

\begin{verbatim}
send :: Message a => Connection -> (Serial -> IO b) -> a
     -> IO (Either MarshalError b)
\end{verbatim}
\begin{verbatim}
receive :: Connection -> IO (Either UnmarshalError ReceivedMessage)
\end{verbatim}

The second parameter to {\tt send} is useful for implementing method-call
semantics -- it will be called before the message is sent to the bus, with
the message's serial, and can be used to perform callback registration.

\section{Types and values}

To provide type-safe conversion between D-Bus and Haskell values, the
{\tt DBus.Types} module defines types and functions for generating and
inspecting D-Bus values. These are split into two categories, the
``atomic'' and ``container'' types.

\subsection{Containers}

\subsubsection{Variants}

{\tt Variant}s can store any type which is an instance of {\tt Variable}.
Clients may implement this class to provide easier marshaling of custom
data types. Any D-Bus value may be converted to a {\tt Variant} via the
{\tt toVariant} function, and converted back using {\tt fromVariant}.

\subsubsection{Arrays}

{\tt Array}s are homogenous, arbitrary-length sequences. To ensure that
all members have the same D-Bus type, {\tt Array}s must be built using
{\tt toArray} or {\tt arrayFromItems}. They may be inspected using
{\tt fromArray} and {\tt arrayItems}.

To provide a more efficient interface for byte arrays, the {\tt arrayToBytes}
and {\tt arrayFromBytes} functions convert between lazy {\tt ByteString}s
and {\tt Array}s. Additionally, both lazy and strict {\tt ByteString}s are
instances of {\tt Variable}.

\subsubsection{Dictionaries}

{\tt Dictionary} is a homogenous, arbitrary-size map. The key type must
be atomic, but the value type may be any valid D-Bus type. Like {\tt Array},
they are accessed via helper functions to enforce type safety.

Dictionaries are constructed using {\tt toDictionary} or
{\tt dictionaryFromItems}, and can be inspected using {\tt fromDictionary}
or {\tt dictionaryItems}.

\subsubsection{Structures}

{\tt Structure}s are heterogeneous, fixed-length sequences, similar to
Haskell tuples. They can be built using the {\tt Structure} constructor, and
don't have any interesting functions.

\subsection{Atomics}

Most D-Bus atomic types map directly to an equivalent Haskell type.
Instances of {\tt Variable} exist for the following Haskell types:

\begin{multicols}{3}
\begin{itemize}{\tt 
\item Bool
\item Double
\item String
\item Text
\item Word8
\item Word16
\item Word32
\item Word64
\item Int16
\item Int32
\item Int64
}
\end{itemize}
\end{multicols}

In addition, there are some specially formatted text types defined in the
D-Bus specification, so they are assigned their own types. Each text type
has functions for parsing and display. Additionally, an ``unsafe'' parsing
function is defined for cases where the input is known in advance to be
valid.

\begin{table}[h]
\begin{tabular}{llll}
\toprule
Type & Display & Parsing & Unsafe parsing \\
\midrule
{\tt Signature} & {\tt strSignature} & {\tt mkSignature} & {\tt mkSignature'} \\
{\tt ObjectPath} & {\tt strObjectPath} & {\tt mkObjectPath} & {\tt mkObjectPath'} \\
{\tt BusName} & {\tt strBusName} & {\tt mkBusName} & {\tt mkBusName'} \\
{\tt InterfaceName} & {\tt strInterfaceName} & {\tt mkInterfaceName} & {\tt mkInterfaceName'} \\
{\tt ErrorName} & {\tt strErrorName} & {\tt mkErrorName} & {\tt mkErrorName'} \\
{\tt MemberName} & {\tt strMemberName} & {\tt mkMemberName} & {\tt mkMemberName'} \\
\bottomrule
\end{tabular}
\end{table}

\subsection{Messages}

Four message types are currently supported -- {\sc method\_call},
{\sc method\_return}, {\sc error}, and {\sc signal}. Each message type has
its own {\tt data} type, defined in {\tt DBus.Message}.

\end{document}
